\section{Работа, мощность, энергия}

\hfill\textit{Лекция от 21.10.2025}

\setcounter{footnote}{-3}

\begin{center}
\begin{tblr}{
  colspec = {X[1.2,c]X[2,c]},
  vlines,
  hlines,
  row{1} = {font=\bfseries,mode=text},
  row{odd} = {bg=gray!10},
  cells = {mode=math,c,m},
  rowsep = 6pt
}
\text{Работа и мощность} & \text{Формулы} \\
\text{Механическая работа} & 
A = \int_L \vec{F}\cdot d\vec{r} = \vec{F}\cdot\vec{s} = Fs\cos\alpha
\\
\text{Мощность} & P = \dfrac{dA}{dt} = \vec{F}\cdot\vec{v} = Fv\cos\alpha \\
\end{tblr}
\end{center}

\begin{center}
\begin{tblr}{
  colspec = {X[1.2,c]X[2,c]},
  vlines,
  hlines,
  row{1} = {font=\bfseries,mode=text},
  row{odd} = {bg=gray!10},
  cells = {c,m},
  rowsep = 6pt
}
\text{Энергия} & \text{Формулы и определения} \\
\begin{tabular}{@{}c@{}}Полная \\ механическая энергия\end{tabular} & 
\begin{tabular}{@{}c@{}}
$E = K + T + U$ \\
$\begin{array}{l}
K - \text{поступательная кинетическая} \\
T - \text{вращательная кинетическая} \\
U - \text{потенциальная энергия}
\end{array}$
\end{tabular}
\\
\begin{tabular}{@{}c@{}}Кинетические энергии\end{tabular} & 
\begin{tabular}{@{}c@{}}
$\displaystyle K = \frac{1}{2}mv^2$ \qquad
$\displaystyle T = \frac{1}{2}I\omega^2$
\end{tabular} \\
\begin{tabular}{@{}c@{}}Закон сохранения \\ механической энергии\end{tabular} & 
\begin{tabular}{@{}c@{}}
$E_1 = E_2$ \\
для консервативных систем
\end{tabular} \\
\begin{tabular}{@{}c@{}}Консервативные силы\end{tabular} & 
\begin{tabular}{@{}c@{}}
$\displaystyle \vec{F} = -\nabla U=-\grad U$\footnotemark \\
$\displaystyle A_{1\to2} = U_1 - U_2$
\end{tabular} \\
\begin{tabular}{@{}c@{}}Примеры \\ потенциальных \\ энергий\end{tabular} & 
\begin{tabular}{@{}c@{}}
Гравитация: $U = mgh$ \\
Упругость: $U = \dfrac{1}{2}kx^2$ \\
Кулоновская: $U = \dfrac{kq_1q_2}{r}$
\end{tabular} \\
\end{tblr}
\end{center}

\footnotetext{$\nabla$ -- оператор набла (градиент): $\nabla = \left(\dfrac{\partial}{\partial x}, \dfrac{\partial}{\partial y}, \dfrac{\partial}{\partial z}\right)=\dfrac{\partial}{\partial x}\vec{i}+\dfrac{\partial }{\partial y}\vec{j}+\dfrac{\partial }{\partial z}\vec{k}$}