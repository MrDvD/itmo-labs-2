\section{Системы координат}

\hfill\textit{Лекция от 02.09.2025}

\begin{center}
\begin{tblr}{
  colspec = {X[c]X[c]},
  vlines,
  hlines,
  row{1} = {font=\bfseries,mode=text},
  row{odd} = {bg=gray!10},
  cells = {c,m},
  rowsep = 6pt
}
\text{Левая система координат} & \text{Правая система координат} \\
\begin{tikzpicture}[scale=1.0, >=stealth]
  %--- координаты ---
  \coordinate (O) at (0,0);
  % Левая система: ось x направлена вправо, ось y "вглубь" (влево-вниз)
  \coordinate (X) at (2.2,0);
  \coordinate (Y) at (-1.2,-1.4);
  \coordinate (Z) at (0,2.0);

  %--- положительные оси ---
  \draw[line width=0.9pt,->] (O) -- (X) node[below right,font=\small] {$x$};
  \draw[line width=0.9pt,->] (O) -- (Y) node[below left,font=\small] {$y$};
  \draw[line width=0.9pt,->] (O) -- (Z) node[above,font=\small] {$z$};

  %--- отрицательные направления ---
  \draw[dashed] (O) -- ($(O)!-0.6!(X)$);
  \draw[dashed] (O) -- ($(O)!-0.6!(Y)$);
  \draw[dashed] (O) -- ($(O)!-0.5!(Z)$);

  %--- метка начала ---
  \node[font=\small] at ($(O)+(0.15,-0.2)$) {$O$};
\end{tikzpicture}
&
\begin{tikzpicture}[scale=1.0, >=stealth]
  %--- координаты ---
  \coordinate (O) at (0,0);
  % Левая система: ось x направлена вправо, ось y "вглубь" (влево-вниз)
  \coordinate (Y) at (2.2,0);
  \coordinate (X) at (-1.2,-1.4);
  \coordinate (Z) at (0,2.0);

  %--- положительные оси ---
  \draw[line width=0.9pt,->] (O) -- (X) node[below left,font=\small] {$x$};
  \draw[line width=0.9pt,->] (O) -- (Y) node[below right,font=\small] {$y$};
  \draw[line width=0.9pt,->] (O) -- (Z) node[above,font=\small] {$z$};

  %--- отрицательные направления ---
  \draw[dashed] (O) -- ($(O)!-0.6!(X)$);
  \draw[dashed] (O) -- ($(O)!-0.6!(Y)$);
  \draw[dashed] (O) -- ($(O)!-0.5!(Z)$);

  %--- метка начала ---
  \node[font=\small] at ($(O)+(0.15,-0.2)$) {$O$};
\end{tikzpicture}
\\
Векторное произведение: & Векторное произведение: \\
$\vec{x} \times \vec{y} = -\vec{z}$ & $\vec{x} \times \vec{y} = \vec{z}$ \\
Используется редко & Стандарт в нашем курсе \\
\end{tblr}
\end{center}

\begin{table}[ht]
\begin{tblr}{
  colspec = {X[1,c]X[2,c]X[1,c]},
  vlines,
  hlines,
  row{1} = {font=\bfseries,mode=text},
  row{odd} = {bg=gray!10},
  cells = {mode=math,c,m},
  rowsep = 6pt
}
\text{Система} & \text{Координаты} & \text{Формулы связи} \\
\text{Декартова} & 
(x, y, z) & 
--- \\
\text{Полярная} & 
\begin{array}{c} (r, \varphi) \\ \text{радиус, полярный угол} \end{array} & 
\begin{array}{l}
x = r\cos\varphi \\
y = r\sin\varphi
\end{array} \\
\text{Цилиндрическая} & 
\begin{array}{c} (\rho, \varphi, z) \\ \text{радиус, полярный угол, высота} \end{array} & 
\begin{array}{l}
x = \rho\cos\varphi \\
y = \rho\sin\varphi \\
z = z
\end{array} \\
\text{Сферическая} & 
\begin{array}{c} (r, \theta, \varphi) \\ \text{радиус, зенит, азимут} \end{array} & 
\begin{array}{l}
x = r\sin\theta\cos\varphi \\
y = r\sin\theta\sin\varphi \\
z = r\cos\theta
\end{array} \\
\end{tblr}
\caption{Связи координат в различных системах}
\end{table}