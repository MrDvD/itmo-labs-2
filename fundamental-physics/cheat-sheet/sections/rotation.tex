\section{Вращательное движение твёрдого тела}

\begin{table}[ht]
\centering
\begin{tblr}{
  colspec = {XX},
  vlines,
  hlines,
  row{1} = {font=\bfseries},
  row{odd} = {bg=gray!10},
  cells = {mode=math,c},
  rowsep = 6pt
}
\text{Поступательное движение} & \text{Вращательное движение} \\
\vec{r} & \alpha \\
\dfrac{\diff\vec{r}}{\diff t} = \vec{v}\quad\left[\dfrac{\text{м}}{\text{с}}\right] & \dfrac{\diff\vec{\alpha}}{\diff t} = \vec{\omega}\quad\left[\dfrac{\text{рад}}{\text{с}}\right] \\
\begin{array}{c} \vec{f} \\ \text{сила} \end{array} & \begin{array}{c} \vec{M}=[\vec{r}\times\vec{f}] \\ \text{момент силы} \end{array} \\
\begin{array}{c} m \\ \text{масса} \end{array} & \begin{array}{c} I \\ \text{момент инерции} \end{array} \\
\begin{array}{c} \vec{p} = m \vec{v} \\ \text{импульс} \end{array} & \begin{array}{c} \vec{L} = I \vec{\omega} \\ \text{момент импульса} \end{array} \\
\dfrac{\diff \vec{p}}{\diff t} = \vec{f} & \dfrac{\diff \vec{L}}{\diff t} = \vec{M} \\
\end{tblr}
\caption{Сравнение параметров движения}
\end{table}

\begin{table}[ht]
\centering
\begin{tblr}{
  colspec = {X[1.7,c]X[3,c]},
  vlines,
  hlines,
  row{1} = {font=\bfseries},
  row{odd} = {bg=gray!10},
  cells = {mode=math,c,m},
  rowsep = 8pt
}
\text{Тип системы} & \text{Формула момента инерции} \\
\text{Точечные массы} & I = \sum_{i} m_i r_i^2 = m_1r_1^2 + m_2r_2^2 + \cdots + m_nr_n^2 \\
\text{Непрерывное тело} & I = \int r^2  dm = \int_V \rho(\vec{r}) r^2  dV \\
\text{Линейная плотность} & I = \int r^2  dm = \int_L \lambda(l) r^2  dl,\quad dm = \lambda  dl \\
\text{Поверхностная плотность} & I = \int r^2  dm = \int_S \sigma(\vec{r}) r^2  dS,\quad dm = \sigma  dS \\
\end{tblr}
\caption{Общие формулы для вычисления момента инерции}
\end{table}

\begin{center}
\begin{tblr}{
  colspec = {XX},
  vlines,
  hlines,
  row{1} = {font=\bfseries},
  row{odd} = {bg=gray!10},
  row{2} = {valign=m},
  cells = {mode=math,c},
  rowsep = 6pt
}
\text{Теорема Гюйгенса-Штейнера} & \text{Схема} \\
I_{\text{new}} = I_{\text{ц.м.}} + md^2 
&
\begin{minipage}[c]{\linewidth}
\centering
\begin{tikzpicture}[scale=1.0]
    % Основные координаты
    \coordinate (B) at (1.5,0.8); % Центр масс
    \coordinate (A) at (3,1.5);   % Новая ось
    
    % Тело (произвольная фигура)
    \draw[fill=black!10, draw=black, thick] (B) circle (0.8);
    
    % Центр масс
    \filldraw[black] (B) circle (2pt) node[below right] {Ц.М.};
    
    % Новая ось
    \filldraw[black] (A) circle (2pt) node[above right] {Новая ось};
    
    % Расстояние d
    \draw[dashed, thick] (B) -- (A) node[midway, above] {$d$};
    
    % Обозначения моментов инерции
    \node at (1.5,2.2) {$I_{\text{ц.м.}}$};
    \node at (3.5,2.2) {$I_{\text{new}}$};
    
    % Стрелки к осям
    \draw[->, thick] (1.8,2) to[out=0,in=90] (B);
    \draw[->, thick] (3.2,2) to[out=180,in=90] (A);
    
    % Масса
    \node at (1,1.8) {$m$};
\end{tikzpicture}
\end{minipage}
\\
\end{tblr}
\end{center}

\begin{table}[H]
\centering
\begin{tblr}{
  colspec = {lcc},
  vlines,
  hlines,
  row{1} = {font=\bfseries,mode=text},
  row{odd} = {bg=gray!10},
  cells = {mode=math,c,m},
  rowsep = 4pt
}
\text{Тело} & \text{Ось вращения} & \text{Момент инерции} \\
\text{Стержень} & \text{Через центр} & \dfrac{1}{12}mL^2 \\
\text{Стержень} & \text{Через конец} & \dfrac{1}{3}mL^2 \\
\text{Кольцо} & \text{Через центр} & mR^2 \\
\text{Диск} & \text{Через центр} & \dfrac{1}{2}mR^2 \\
\text{Шар} & \text{Через центр} & \dfrac{2}{5}mR^2 \\
\text{Сфера} & \text{Через центр} & \dfrac{2}{3}mR^2 \\
\text{Цилиндр} & \text{Ось симметрии} & \dfrac{1}{2}mR^2 \\
\text{Пластина } a\times b & \text{Через центр} & \dfrac{1}{12}m(a^2 + b^2) \\
\end{tblr}
\caption{Моменты инерции однородных тел}
\end{table}