\begin{problem}
  Дано векторное поле \( \vec{a} \) и плоскость \( \sigma \), пересекающая координатные плоскости по замкнутой ломаной \( KLMK \), где \( K \), \( L \), \( M \) --- точки пересечения плоскости \( \sigma \) координатными осями \( Ox \), \( Oy \), \( Oz \) соответственно:
  %
  \[
    \vec{a}=(3z-x)\vec{i}+(z-y)\vec{j},\quad\sigma\colon x-2y-z=-4
  \]
  \begin{enumerate}[(a)]
    \item Найдите поток \( Q \) векторного поля \( \vec{a} \) через часть \( S \) плоскости \( \sigma \), вырезанной координатными плоскостями, в сторону нормали \( \vec{n} \), направленной от начала координат \( O(0, 0, 0) \).
    \item С помощью теоремы Остроградского-Гаусса найдите поток \( Q \) векторного поля \( \vec{a} \) через полную поверхность тетраэдра \( OLMK \) в сторону внешней нормали.
    \item Найдите циркуляцию \( C \) векторного поля \( \vec{a} \) по контуру \( KLMK \), образованному пересечением плоскости \( \sigma \) с координатными плоскостями.
  \end{enumerate}
\end{problem}

\begin{solution}
  \dots
\end{solution}