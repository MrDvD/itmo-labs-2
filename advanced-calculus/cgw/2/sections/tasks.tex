\section{Список задач}

В рамках выполнения расчётно-графической работы \textit{по варианту №10} необходимо предоставить решения следующих задач:

\begin{enumerate}
  \item Плоская область \( D \) ограничена заданными линиями:
    \[
      y=\sqrt[3]{x},\quad x=-\sqrt{y+2},\quad y=3x-2
    \]
    \begin{enumerate}
      \item Сделайте схематический рисунок области \( D \).
      \item С помощью двойного интеграла найдите площадь области \( D \).
    \end{enumerate}
  \item Тело \( T \) ограничено заданными поверхностями:
    \[
      z=4-x^2-y^2,\quad 9z=-5(x^2+y^2),\quad y=0\text{ при }y\leq 0
    \]
    \begin{enumerate}
      \item Сделайте схематический рисунок тела \( T \).
      \item С помощью тройного интеграла найдите объём тела \( T \), перейдя к цилиндрическим или сферическим координатам.
    \end{enumerate}
  \item С помощью криволинейного интеграла первого рода найдите массу \( M \) дуги плоской материальной кривой, заданной уравнениями:
    \begin{enumerate}
      \item \( \displaystyle y=\frac{1}{x},\quad \sqrt[4]{3}\leq x\leq\sqrt[4]{8},\quad\rho(x,y)=\frac{6x^3}{y^2} \)
      \item \( \begin{cases*}
        &x=2(t-\sin t)\\
        &y=2(1-\cos t)
      \end{cases*},\quad 0\leq t\leq\pi,\quad\rho(x,y)=1 \)
    \end{enumerate}
  \item С помощью криволинейного интеграла первого рода найдите массу \( M \) дуги пространственной материальной кривой, заданной уравнениями:
    \[
      \begin{cases*}
        &x=2t\\
        &y=t\\
        &z=2t+1
      \end{cases*},\quad 6\leq t\leq 20,\quad\rho(x,y,z)=\frac{1}{\sqrt{z^2-x^2}} 
    \]
  \item Дано векторное поле \( \vec{a} \) и плоскость \( \sigma \), пересекающая координатные плоскости по замкнутой ломаной \( KLMK \), где \( K \), \( L \), \( M \) --- точки пересечения плоскости \( \sigma \) координатными осями \( Ox \), \( Oy \), \( Oz \) соответственно:
    %
    \[
      \vec{a}=(3z-x)\vec{i}+(z-y)\vec{j},\quad\sigma\colon x-2y-z=-4
    \]
    \begin{enumerate}
      \item Найдите поток \( Q \) векторного поля \( \vec{a} \) через часть \( S \) плоскости \( \sigma \), вырезанной координатными плоскостями, в сторону нормали \( \vec{n} \), направленной от начала координат \( O(0, 0, 0) \).
      \item С помощью теоремы Остроградского-Гаусса найдите поток \( Q \) векторного поля \( \vec{a} \) через полную поверхность тетраэдра \( OLMK \) в сторону внешней нормали.
      \item Найдите циркуляцию \( C \) векторного поля \( \vec{a} \) по контуру \( KLMK \), образованному пересечением плоскости \( \sigma \) с координатными плоскостями.
    \end{enumerate}
  \item Дано векторное поле \( \vec{a}(M) \):
    \[
      \vec{a}=(1+e^y)\vec{i}+(xe^y-e^z-1)\vec{j}+(2-ye^z)\vec{k}
    \]
    \begin{enumerate}
      \item Проверьте, является ли векторное поле соленоидальным или потенциальным.
      \item Если поле потенциально, найдите его потенциал.
    \end{enumerate}
\end{enumerate}

\newpage

\section{Решения задач}

\begin{problem}
Вычислить площадь фигуры, ограниченной кривыми:
\[
  (x^2+y^2)^2=a(x^3-3xy^2),\quad a> 0
\]
\end{problem}

\begin{solution}
  \dots
\end{solution}
\begin{problem}
Показать, что данное выражение является полным дифференциалом функции \( u(x, y) \). Найти функцию \( u(x, y) \):
\[
  (ye^{xy}+y^2)\diff x+(xe^{xy}+2xy)\diff y
\]
\end{problem}

\begin{solution}
  Проверим условие Грина:
  %
  \[
    \begin{gathered}
      \begin{cases*}
        &\frac{\partial(ye^{xy}+y^2)}{\partial y}=e^{xy}+xye^{xy}+2y\\
        &\frac{\partial(xe^{xy}+2xy)}{\partial x}=e^{xy}+xye^{xy}+2y
      \end{cases*}\implies\dfrac{\partial(ye^{xy}+y^2)}{\partial y}=\dfrac{\partial(xe^{xy}+2xy)}{\partial x}\implies\\
      \implies\exists u(x,y)\colon\diff u(x,y)=(ye^{xy}+y^2)\diff x+(xe^{xy}+2xy)\diff y
    \end{gathered}
  \]

  Найдём функцию \( u(x,y) \) из определения полного дифференциала:
  %
  \[
    \dfrac{\partial u}{\partial x}(x,y)=ye^{xy}+y^2\implies u(x,y)=\int (ye^{xy}+y^2)\diff x=e^{xy}+xy^2+C(y)
  \]

  Найдём функцию \( C(y) \):
  %
  \[
    \begin{cases*}
      &\frac{\partial u}{\partial y}(x,y)=xe^{xy}+2xy\\
      &\frac{\partial u}{\partial y}(x,y)=xe^{xy}+2xy+C'(y)
    \end{cases*}\implies
    C'(y)=0\implies C(y)=C\in\mathbb{R}
  \]

  Итого:
  %
  \[
    \boxed{
      u(x,y)=e^{xy}+xy^2+C,\quad C\in\mathbb{R}
    }
  \]
\end{solution}
\begin{problem}
  Найти аналитическую функцию по известной её действительной части:
  %
  \[ u(x,y)=\sh(2x)\cos(2y)+x^2-y^2+4y-4 \]
\end{problem}

\begin{solution}
  \dots
\end{solution}
\begin{problem}
  С помощью криволинейного интеграла первого рода найдите массу \( M \) дуги пространственной материальной кривой, заданной уравнениями:
  \[
    \begin{cases*}
      &x=2t\\
      &y=t\\
      &z=2t+1
    \end{cases*},\quad 6\leq t\leq 20,\quad\rho(x,y,z)=\frac{1}{\sqrt{z^2-x^2}} 
  \]
\end{problem}

\begin{solution}
  Найдём \( \diff l \):
  %
  \[
    \diff l=\sqrt{(x_t')^2+(y_t')^2+(z_t')^2}\diff t=\sqrt{4+1+4}\diff t=3\diff t
  \]

  Найдём массу \( M \) дуги пространственной кривой \( L \):
  %
  \[
    \begin{gathered}
      M=\int_L\rho(x,y,z)\diff l=3\int_6^{20}\dfrac{\diff t}{\sqrt{(2t+1)^2-4t^2}}=3\int_6^{20}\dfrac{\diff t}{\sqrt{4t+1}}=\\
      =\left[\diff(4t+1)=4\diff t\right]=\dfrac{3}{4}\int_6^{20}\dfrac{\diff(4t+1)}{\sqrt{4t+1}}=\left.\dfrac{3}{2}\sqrt{4t+1}\right|_6^{20}=\\
      =\dfrac{3}{2}(9-5)=6
    \end{gathered}
  \]

  Итого:
  %
  \[
    \boxed{
      M=6
    }
  \]
\end{solution}
\begin{problem}
Заменяя приращение функции дифференциалом, приближённо вычислить:
\[ \sin 88^{\circ}\cdot\tg 46^{\circ} \]
\end{problem}

\begin{solution}
  Введём функцию \( f(x,y)=\sin x\cdot\tg y \). Она дифференцируема в точке \( M(88^{\circ},46^{\circ}) \) как произведение дифференцируемых функций.

  По определению дифференцируемости функции \( f \):
  %
  \[ f(x_0+\Delta x,y_0+\Delta y)-f(x_0,y_0)=\diff f(x_0,y_0)(\Delta x,\Delta y)+\underset{(\Delta x,\Delta y)\to (0,0)}{o\left(\sqrt{\Delta x^2+\Delta y^2}\right)} \]
  
  Отбросим нелинейную часть приращения функции:
  %
  \[ f(x_0+\Delta x,y_0+\Delta y)\approx f(x_0,y_0)+\diff f(x_0,y_0)(\Delta x,\Delta y) \]
  
  По определению дифференциала функции \( f \):
  %
  \[ \diff f(x_0,y_0)(\Delta x,\Delta y)=\dfrac{\partial f}{\partial x}\left(x_0,y_0\right)\diff x(\Delta x)+\dfrac{\partial f}{\partial y}\left(x_0,y_0\right)\diff y(\Delta y) \]
  
  Полагая для нашей задачи \( M_0\left(\frac{\pi}{2},\frac{\pi}{4}\right) \), \( \Delta\vec{r}=\left(-\frac{\pi}{90},\frac{\pi}{180}\right) \), найдём частные производные в точке \( M_0 \):
  %
  \begin{align*}
  &\frac{\partial f}{\partial x}(M_0) = \cos x_0 \cdot \tan y_0 = \cos\frac{\pi}{2} \cdot \tan\frac{\pi}{4} = 0 \cdot 1 = 0 \\
  &\frac{\partial f}{\partial y}(M_0) = \frac{\sin x_0}{\cos^2 y_0} = \frac{\sin\frac{\pi}{2}}{\cos^2\frac{\pi}{4}}= 2
  \end{align*}
  
  Найдём значение дифференциала функции \( f \):
  %
  \[ \diff f(M_0)(\Delta\vec{r})=0\cdot\left(-\dfrac{\pi}{90}\right)+2\cdot\dfrac{\pi}{180}=\dfrac{\pi}{90} \]
  
  Найдём значение функции \( f \) в точке \( M_0 \):
  %
  \[ f(M_0)=\sin\frac{\pi}{2}\cdot\tg\frac{\pi}{4}=1 \]
  
  Итого имеем:
  %
  \[ f(M)\approx 1+\dfrac{\pi}{90}\approx 1.034907 \]
  
  Определим относительную погрешность вычислений:
  %
  \[ f(M)=1.034899\ldots\implies\varepsilon=\dfrac{0.000008}{1.034899}\cdot 100\%\approx 7.7\cdot 10^{-4}\,\% \]
\end{solution}
\begin{problem}
Найти наибольшее и наименьшее значения функции \( z = z(x, y) \) в области \( D \), ограниченной заданными линиями:
\[ z = x^{2} - 2xy + \frac{5}{2}y^2 - 2x, \quad D:\ x=0,\ x=2,\ y=0,\ y=2 \]
\end{problem}

\begin{solution}
  \dots
\end{solution}