\begin{problem}
Заменяя приращение функции дифференциалом, приближённо вычислить:
\[ \sin 88^{\circ}\cdot\tg 46^{\circ} \]
\end{problem}

\begin{solution}
  Введём функцию \( f(x,y)=\sin x\cdot\tg y \). Она дифференцируема в точке \( M(88^{\circ},46^{\circ}) \) как произведение дифференцируемых функций.

  По определению дифференцируемости функции \( f \):
  %
  \[ f(x_0+\Delta x,y_0+\Delta y)-f(x_0,y_0)=\diff f(x_0,y_0)(\Delta x,\Delta y)+\underset{(\Delta x,\Delta y)\to (0,0)}{o\left(\sqrt{\Delta x^2+\Delta y^2}\right)} \]
  %
  Отбросим нелинейную часть приращения функции:
  %
  \[ f(x_0+\Delta x,y_0+\Delta y)\approx f(x_0,y_0)+\diff f(x_0,y_0)(\Delta x,\Delta y) \]
  %
  По определению дифференциала функции \( f \):
  %
  \[ \diff f(x_0,y_0)(\Delta x,\Delta y)=\dfrac{\partial f}{\partial x}\left(x_0,y_0\right)\diff x(\Delta x)+\dfrac{\partial f}{\partial y}\left(x_0,y_0\right)\diff y(\Delta y) \]
  %
  Полагая для нашей задачи \( M_0\left(\frac{\pi}{2},\frac{\pi}{4}\right) \), \( \Delta\vec{r}=\left(-\frac{\pi}{90},\frac{\pi}{180}\right) \), найдём частные производные в точке \( M_0 \):
  %
  \begin{align*}
  &\frac{\partial f}{\partial x}(M_0) = \cos x_0 \cdot \tan y_0 = \cos\frac{\pi}{2} \cdot \tan\frac{\pi}{4} = 0 \cdot 1 = 0 \\
  &\frac{\partial f}{\partial y}(M_0) = \frac{\sin x_0}{\cos^2 y_0} = \frac{\sin\frac{\pi}{2}}{\cos^2\frac{\pi}{4}}= 2
  \end{align*}
  %
  Найдём значение дифференциала функции \( f \):
  %
  \[ \diff f(M_0)(\Delta\vec{r})=0\cdot\left(-\dfrac{\pi}{90}\right)+2\cdot\dfrac{\pi}{180}=\dfrac{\pi}{90} \]
  %
  Найдём значение функции \( f \) в точке \( M_0 \):
  %
  \[ f(M_0)=\sin\frac{\pi}{2}\cdot\tg\frac{\pi}{4}=1 \]
  %
  Итого имеем:
  %
  \[ f(M)\approx 1+\dfrac{\pi}{90}\approx 1.034907 \]
  %
  Определим относительную погрешность вычислений:
  %
  \[ f(M)=1.034899\ldots\implies\varepsilon=\dfrac{0.000008}{1.034899}\cdot 100\%\approx 7.7\cdot 10^{-4}\,\% \]
\end{solution}