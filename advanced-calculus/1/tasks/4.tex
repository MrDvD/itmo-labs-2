\begin{problem}
Найдите угол между градиентами функций \( u(x,y,z) \), \( v(x,y,z) \) в точке \( M_0 \):
\[ v = \frac{3x^2}{\sqrt{2}} - \frac{y^2}{\sqrt{2}} + \sqrt{2}z^{2}, \quad u = \frac{z^2}{x^2y^2}, \quad M_0\left(\frac{2}{3};2;\sqrt{\frac{2}{3}}\right) \]
\end{problem}

\begin{solution}
  По определению градиента функций \( u(x,y,z) \), \( v(x,y,z) \):
  %
  \[ \grad u=\left(\dfrac{\partial u}{\partial x},\dfrac{\partial u}{\partial y},\dfrac{\partial u}{\partial z}\right)\qquad\grad v=\left(\dfrac{\partial v}{\partial x},\dfrac{\partial v}{\partial y},\dfrac{\partial v}{\partial z}\right) \]
  %
  По определению скалярного произведения векторов:
  %
  \[
  \grad u\cdot\grad v = \abs{\grad u}\cdot\abs{\grad v}\cdot\cos\alpha, \quad \alpha = \angle(\grad u, \grad v)
  \]
  \begin{equation}
  \cos\alpha = \dfrac{\grad u\cdot\grad v}{\abs{\grad u}\cdot\abs{\grad v}} \label{eq:cos_alpha}
  \end{equation}
  %
  Перепишем скалярное произведение векторов в координатной форме:
  %
  \begin{equation}
  \grad u\cdot\grad v =
  \begin{pmatrix}
  \dfrac{\partial u}{\partial x} & \dfrac{\partial u}{\partial y} & \dfrac{\partial u}{\partial z}
  \end{pmatrix}
  \cdot
  \begin{pmatrix}
  \dfrac{\partial v}{\partial x} \\[2ex]
  \dfrac{\partial v}{\partial y} \\[2ex]
  \dfrac{\partial v}{\partial z}
  \end{pmatrix}
  =
  \frac{\partial u}{\partial x}\frac{\partial v}{\partial x}
  + \frac{\partial u}{\partial y}\frac{\partial v}{\partial y}
  + \frac{\partial u}{\partial z}\frac{\partial v}{\partial z}
  \label{eq:grad_dot_product}
  \end{equation}
  %
  Вычислим частные производные функций \( u(x,y,z) \), \( v(x,y,z) \):
  %
  \[
  \begin{array}{rcl}
  \dfrac{\partial u}{\partial x} & = & -\dfrac{2z^2}{x^3y^2} \\[2ex]
  \dfrac{\partial u}{\partial y} & = & -\dfrac{2z^2}{x^2y^3} \\[2ex]
  \dfrac{\partial u}{\partial z} & = & \dfrac{2z}{x^2y^2}
  \end{array}
  \quad\quad
  \begin{array}{rcl}
  \dfrac{\partial v}{\partial x} & = & 3\sqrt{2}x \\[2ex]
  \dfrac{\partial v}{\partial y} & = & -\sqrt{2}y \\[2ex]
  \dfrac{\partial v}{\partial z} & = & 2\sqrt{2}z
  \end{array}
  \]
  %
  Подставим полученные частные производные в сумму \eqref{eq:grad_dot_product}:
  %
  \[
  \begin{aligned}
  \grad u\cdot\grad v &= -\dfrac{2z^2}{x^3y^2}\cdot 3\sqrt{2}x
                        -\dfrac{2z^2}{x^2y^3}\cdot(-\sqrt{2}y)
                        +\dfrac{2z}{x^2y^2}\cdot 2\sqrt{2}z \\
                      &= \left(-6\sqrt{2}+2\sqrt{2}+4\sqrt{2}\right)\cdot\dfrac{z^2}{x^2y^2} = 0
  \end{aligned}
  \]
  %
  Учитывая, что модули градиентов функций \( u(x,y,z) \), \( v(x,y,z) \) тождественно не равны нулю, из \eqref{eq:cos_alpha} получаем:
  %
  \[ \cos\alpha=0\implies\alpha=\dfrac{\pi}{2} \]
  %
  Таким образом, градиенты функций \( u(x,y,z) \), \( v(x,y,z) \) взаимно перпендикулярны независимо от выбора точки \( M_0 \).
\end{solution}