\begin{problem}
Найти наибольшее и наименьшее значения функции \( z = z(x, y) \) в области \( D \), ограниченной заданными линиями:
\[ z = x^{2} - 2xy + \frac{5}{2}y^2 - 2x, \quad D:\ x=0,\ x=2,\ y=0,\ y=2 \]
\end{problem}

\begin{solution}
  Наибольшее и наименьшее значение располагается в экстремумах и/или на границах рассматриваемой области \( D \).

  По необходимому условию экстремума в точке \( M(x,y) \):
  %
  \[ \left(\dfrac{\partial z}{\partial x}(M)=0\right)\land\left(\dfrac{\partial z}{\partial y}(M)=0\right) \]
  
  Вычислим частные производные функции \( z(x,y) \):
  %
  \[ \dfrac{\partial z}{\partial x}=2x-2y-2\qquad\dfrac{\partial z}{\partial y}=-2x+5y \]
  
  Найдём стационарные точки функции \( z(x,y) \):
  %
  \[
  \begin{cases*}
    &2x-2y-2=0\\
    &-2x+5y=0
  \end{cases*}\implies
  M_0\left(\frac{5}{3},\frac{2}{3}\right)\in D\text{ --- стационарная}
  \]
  
  По достаточному условию экстремума в стационарной точке \( M_0 \):
  %
  \[ \Delta=
  \begin{vmatrix}
    \dfrac{\partial^2 z}{\partial x^2}(M_0) & \dfrac{\partial^2 z}{\partial x\partial y}(M_0) \\[2ex]
    \dfrac{\partial^2 z}{\partial y\partial x}(M_0) & \dfrac{\partial^2 z}{\partial y^2}(M_0)
  \end{vmatrix}\neq 0
  \]
  
  Вычислим частные производные второго порядка функции \( z(x,y) \):
  %
  \[ \dfrac{\partial^2 z}{\partial x^2}=2\qquad\dfrac{\partial^2 z}{\partial x\partial y}=-2\qquad\dfrac{\partial^2 z}{\partial y\partial z}=-2\qquad\dfrac{\partial^2 z}{\partial y^2}=5 \]
  
  Полученные частные производные постоянны, значит, Гессиан функции \( z(x,y) \) не зависит от выбора стационарной точки и равен:
  %
  \[ \Delta=
  \begin{vmatrix}
    2 & -2\\
    -2 & 5
  \end{vmatrix}=2\cdot 5-(-2)\cdot(-2)=6>0\implies M_0\text{ --- локальный минимум}
  \]
  
  Найдём значение функции \( z(x,y) \) в точке \( M_0 \):
  %
  \[ z\left(\dfrac{5}{3},\dfrac{2}{3}\right)=\left(\dfrac{5}{3}\right)^2-2\cdot\dfrac{5}{3}\cdot\dfrac{2}{3}+\dfrac{5}{2}\cdot\left(\dfrac{2}{3}\right)^2-2\cdot\dfrac{5}{3}=\dfrac{25-20+10-30}{9}=-\dfrac{5}{3} \]
  
  Значит, функция \( z(x,y) \) принимает своё \textbf{наименьшее значение} \( -\frac{5}{3} \) в точке \( M_0\in D \). Для поиска максимального значения рассмотрим поведение функции на границах области \( D \):
  %
  \begin{enumerate}
    \item \( \measuredangle\, x=0\implies z(y)=\frac{5}{2}y^2 \) --- парабола с растянутыми ветвями вверх и вершиной в \( (0,0) \).
    
    Максимум достигается при \( y=2\implies z(2)=10 \).
    \item \( \measuredangle\, x=2\implies z(y)=4-4y+\frac{5}{2}y^2-4=\frac{5}{2}y^2-4y=\frac{5}{2}\left(y-\frac{4}{5}\right)^2-\frac{8}{5} \) --- парабола с растянутыми ветвями вверх и вершиной в \( (\frac{4}{5},-\frac{8}{5}) \).
    
    Максимум достигается при \( y=2\implies z(2)=2 \).
    \item \( \measuredangle\, y=0\implies z(x)=x^2-2x=(x-1)^2-1 \) --- парабола с ветвями вверх и вершиной в \( (1,-1) \).
    
    Максимум достигается при \( x=0 \) и \( x=2\implies z(0)=z(2)=0 \).
    \item \( \measuredangle\, y=2\implies z(x)=x^2-4x+10-2x=(x-3)^2+1 \) --- парабола с ветвями вверх и вершиной в \( (3,1) \).
    
    Максимум достигается при \( x=0\implies z(0)=10 \).
  \end{enumerate}
  
  Значит, функция \( z(x,y) \) принимает своё \textbf{наибольшее значение} \( 10 \) в точке \( (0,2)\in D \).
\end{solution}