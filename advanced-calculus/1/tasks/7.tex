\begin{problem}
Найти касательную плоскость и нормаль к поверхности:
\[ x^2 + y^2 + z^2 - 6y + 4z + 4 = 0 \]
в точке $M_0(2, 1, -1)$.
\end{problem}

\begin{solution}
Сначала проверим, что точка лежит на поверхности. Подставляем координаты:
\[ 2^2 + 1^2 + (-1)^2 - 6\cdot1 + 4\cdot(-1) + 4 = 4 + 1 + 1 - 6 - 4 + 4 = 0 \]
Точка действительно лежит на поверхности.

Чтобы найти касательную плоскость, нужно вычислить частные производные. Рассмотрим функцию:
\[ F(x,y,z) = x^2 + y^2 + z^2 - 6y + 4z + 4 \]

Находим производные:
\[ F_x' = 2x,\quad F_y' = 2y - 6,\quad F_z' = 2z + 4 \]

Вычисляем их в точке $M_0(2,1,-1)$:
\[ F_x'(2,1,-1) = 4,\quad F_y'(2,1,-1) = -4,\quad F_z'(2,1,-1) = 2 \]

Эти числа - координаты вектора нормали к поверхности. Уравнение касательной плоскости:
\[ 4(x-2) - 4(y-1) + 2(z+1) = 0 \]

Упрощаем (можно разделить на 2):
\[ 2(x-2) - 2(y-1) + (z+1) = 0 \]
\[ 2x - 4 - 2y + 2 + z + 1 = 0 \]
\[ 2x - 2y + z - 1 = 0 \]

Нормаль - это прямая, проходящая через точку $M_0$ в направлении вектора нормали. Её уравнения:
\[ \frac{x-2}{2} = \frac{y-1}{-2} = \frac{z+1}{1} \]

\textbf{Ответ:}
\begin{itemize}
    \item Касательная плоскость: $2x - 2y + z - 1 = 0$
    \item Нормаль: $\dfrac{x-2}{2} = \dfrac{y-1}{-2} = \dfrac{z+1}{1}$
\end{itemize}
\end{solution}