\begin{problem}
Показать, что данное выражение является полным дифференциалом функции \( u(x, y) \). Найти функцию \( u(x, y) \):
\[
  (ye^{xy}+y^2)\diff x+(xe^{xy}+2xy)\diff y
\]
\end{problem}

\begin{solution}
  Проверим условие Грина:
  %
  \[
    \begin{gathered}
      \begin{cases*}
        &\frac{\partial(ye^{xy}+y^2)}{\partial y}=e^{xy}+xye^{xy}+2y\\
        &\frac{\partial(xe^{xy}+2xy)}{\partial x}=e^{xy}+xye^{xy}+2y
      \end{cases*}\implies\dfrac{\partial(ye^{xy}+y^2)}{\partial y}=\dfrac{\partial(xe^{xy}+2xy)}{\partial x}\implies\\
      \implies\exists u(x,y)\colon\diff u(x,y)=(ye^{xy}+y^2)\diff x+(xe^{xy}+2xy)\diff y
    \end{gathered}
  \]

  Найдём функцию \( u(x,y) \) из определения полного дифференциала:
  %
  \[
    \dfrac{\partial u}{\partial x}(x,y)=ye^{xy}+y^2\implies u(x,y)=\int (ye^{xy}+y^2)\diff x=e^{xy}+xy^2+C(y)
  \]

  Найдём функцию \( C(y) \):
  %
  \[
    \begin{cases*}
      &\frac{\partial u}{\partial y}(x,y)=xe^{xy}+2xy\\
      &\frac{\partial u}{\partial y}(x,y)=xe^{xy}+2xy+C'(y)
    \end{cases*}\implies
    C'(y)=0\implies C(y)=C\in\mathbb{R}
  \]

  Итого:
  %
  \[
    \boxed{
      u(x,y)=e^{xy}+xy^2+C,\quad C\in\mathbb{R}
    }
  \]
\end{solution}