\begin{problem}
Вычислить момент инерции относительно начала координат отрезка
прямой, заключенного между точками \( A(2, 0) \) и \( B(0, 1) \), если линейная
плотность в каждой его точке равна \( 1 \).
\end{problem}

\begin{solution}
  По определению момента инерции относительно точки \( O(0,0) \):
  %
  \[
    I_O=\int_{L_{AB}}\lambda(s)r^2(s)\diff s=[\lambda(s)=1]=\int_{L_{AB}} r^2(s)\diff s
  \]

  Параметризуем отрезок прямой \( L_{AB} \):
  %
  \[
    L_{AB}\colon\begin{cases*}
      &x(t)=-2+2t\\
      &y(t)=t
    \end{cases*},\quad t\in[0,1]
  \]

  Перепишем интеграл:
  %
  \[
    \int_{L_{AB}}r^2(s)\diff s=\int_0^1 r^2(t)\sqrt{(x'(t))^2+(y'(t))^2}\diff t=\sqrt{5}\int_0^1 r^2(t)\diff t
  \]

  Выразим \( r^2(t) \) через \( x=x(t),\ y=y(t) \):
  %
  \[
    r^2(t)=(x-0)^2+(y-0)^2=(2t-2)^2+t^2=5t^2-8t+4
  \]

  Подставим \( r^2(t) \) в последний интеграл:
  %
  \[
    \int_0^1 r^2(t)\diff t=\int_0^1 (5t^2-8t+4)\diff t=\left(\dfrac{5}{3}t^3-4t^2+4t\middle)\right|_0^1=\dfrac{5}{3}
  \]

  Итого:
  %
  \[
    \boxed{
      I_O=\dfrac{5\sqrt{5}}{3}
    }
  \]
\end{solution}