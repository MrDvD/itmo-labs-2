\begin{problem}
Вычислить поверхностный интеграл второго рода:
\[
  \iint_S z^2\diff x\diff y,\quad S\colon\text{внешняя сторона поверхности эллипсоида }x^2+y^2+2z^2=2
\]
\end{problem}

\begin{solution}
  По аддитивности поверхностного интеграла:
  %
  \[ S=S_{\text{верх}}+S_{\text{низ}}\implies\iint_Sz^2\diff x\diff y=\iint_{S_{\text{верх}}}z^2\diff x\diff y+\iint_{S_{\text{низ}}}z^2\diff x\diff y \]

  Параметризуем поверхность \( S_{\text{верх}} \):
  %
  \[
    S_{\text{верх}}\colon
    \begin{cases*}
      &x(\rho,\varphi)=\sqrt{2}\rho\cos\varphi\\
      &y(\rho,\varphi)=\sqrt{2}\rho\sin\varphi\\
      &z(\rho,\varphi)=\sqrt{1-\rho^2}
    \end{cases*},\quad\rho\in\left[0,1\right],\ \varphi\in\left[0,2\pi\right]
  \]

  Параметризуем поверхность \( S_{\text{низ}} \):
  %
  \[
    S_{\text{низ}}\colon
    \begin{cases*}
      &x(\rho,\varphi)=\sqrt{2}\rho\cos\varphi\\
      &y(\rho,\varphi)=\sqrt{2}\rho\sin\varphi\\
      &z(\rho,\varphi)=-\sqrt{1-\rho^2}
    \end{cases*},\quad\rho\in\left[0,1\right],\ \varphi\in\left[0,2\pi\right]
  \]

  Найдём направляющие косинусы для обоих случаев:
  %
  \[
    \cos\gamma_{\text{верх}}=\cos\gamma_{\text{низ}}=\cos\gamma=\begin{vmatrix}
      x'_\rho & y'_\rho \\
      x'_\varphi & y'_\varphi
    \end{vmatrix}=\begin{vmatrix}
      \sqrt{2}\cos\varphi & \sqrt{2}\sin\varphi \\
      -\sqrt{2}\rho\sin\varphi & \sqrt{2}\rho\cos\varphi
    \end{vmatrix}=2\rho
  \]

  Перейдём к кратным интегралам по области \( D \):
  %
  \[
    \iint_{S_{\text{верх}}}z^2\diff x\diff y+\iint_{S_{\text{низ}}}z^2\diff x\diff y=2\iint_D (\sqrt{1-\rho^2}-\sqrt{1-\rho^2})\rho\diff\rho\diff\varphi=0
  \]

  Итого:
  %
  \[
    \boxed{
      \iint_S z^2\diff x\diff y=0
    }
  \]
\end{solution}