\begin{problem}
Вычислить поверхностный интеграл второго рода:
\[
  \iint_S z^2\diff x\diff y,\quad S\colon\text{внешняя сторона поверхности эллипсоида }x^2+y^2+2z^2=2
\]
\end{problem}

\begin{solution}
  По аддитивности поверхностного интеграла:
  %
  \[ S=S_{\text{верх}}+S_{\text{низ}}\implies\iint_Sz^2\diff x\diff y=\iint_{S_{\text{верх}}}z^2\diff x\diff y+\iint_{S_{\text{низ}}}z^2\diff x\diff y \]

  Параметризуем поверхность \( S_{\text{верх}} \):
  %
  \[
    S_{\text{верх}}\colon
    \begin{cases*}
      &x(\rho,\varphi)=\sqrt{2}\rho\cos\varphi\\
      &y(\rho,\varphi)=\sqrt{2}\rho\sin\varphi\\
      &z(\rho,\varphi)=\sqrt{1-\rho^2}
    \end{cases*},\quad\rho\in\left[0,1\right],\ \varphi\in\left[0,2\pi\right]
  \]

  Параметризуем поверхность \( S_{\text{низ}} \):
  %
  \[
    S_{\text{низ}}\colon
    \begin{cases*}
      &x(\rho,\varphi)=\sqrt{2}\rho\cos\varphi\\
      &y(\rho,\varphi)=\sqrt{2}\rho\sin\varphi\\
      &z(\rho,\varphi)=-\sqrt{1-\rho^2}
    \end{cases*},\quad\rho\in\left[0,1\right],\ \varphi\in\left[0,2\pi\right]
  \]

  Найдём нормальные векторы для обоих случаев:
  %
  \[
    \begin{gathered}
      \vec{n}_{\text{верх}}^*=\vec{r}_{\text{верх}}(\rho,\varphi)'_\rho\times\vec{r}_{\text{верх}}(\rho,\varphi)'_\varphi=\begin{vmatrix}
        \vec{i} & \vec{j} & \vec{k} \\
        \sqrt{2}\cos\varphi & \sqrt{2}\sin\varphi & -\dfrac{\rho}{\sqrt{1-\rho^2}} \\
        -\sqrt{2}\rho\sin\varphi & \sqrt{2}\rho\cos\varphi & 0 \\
      \end{vmatrix}=\\
      =\left(\dfrac{\sqrt{2}\rho^2\cos\varphi}{\sqrt{1-\rho^2}},\dfrac{\sqrt{2}\rho^2\sin\varphi}{\sqrt{1-\rho^2}},2\rho\right)\\
      \vec{n}_{\text{низ}}^*=\vec{r}_{\text{низ}}(\rho,\varphi)'_\rho\times\vec{r}_{\text{низ}}(\rho,\varphi)'_\varphi=\begin{vmatrix}
        \vec{i} & \vec{j} & \vec{k} \\
        \sqrt{2}\cos\varphi & \sqrt{2}\sin\varphi & \dfrac{\rho}{\sqrt{1-\rho^2}} \\
        -\sqrt{2}\rho\sin\varphi & \sqrt{2}\rho\cos\varphi & 0 \\
      \end{vmatrix}=\\
      =\left(-\dfrac{\sqrt{2}\rho^2\cos\varphi}{\sqrt{1-\rho^2}},-\dfrac{\sqrt{2}\rho^2\sin\varphi}{\sqrt{1-\rho^2}},2\rho\right)\\
    \end{gathered}
  \]

  Уточним направление нормальных векторов:
  %
  \[
    \grad S=(2x,2y,4z)\implies
    \begin{cases*}
      &\vec{n}_{\text{верх}}=\left(\dfrac{\sqrt{2}\rho^2\cos\varphi}{\sqrt{1-\rho^2}},\dfrac{\sqrt{2}\rho^2\sin\varphi}{\sqrt{1-\rho^2}},2\rho\right)\\
      &\vec{n}_{\text{низ}}=\left(\dfrac{\sqrt{2}\rho^2\cos\varphi}{\sqrt{1-\rho^2}},\dfrac{\sqrt{2}\rho^2\sin\varphi}{\sqrt{1-\rho^2}},-2\rho\right)\\
    \end{cases*}
  \]

  Перейдём к кратным интегралам по области \( D \) из параметризации:
  %
  \[
    \iint_{S_{\text{верх}}}z^2\diff x\diff y+\iint_{S_{\text{низ}}}z^2\diff x\diff y=2\iint_D (1-\rho^2-(1-\rho^2))\rho\diff\sigma=0
  \]

  Итого:
  %
  \[
    \boxed{
      \iint_S z^2\diff x\diff y=0
    }
  \]
\end{solution}