\begin{problem}
Дана функция \( u(M)=u(x,y,z) \) и точки \( M_1 \) и \( M_2 \). Вычислить \( \grad u(M_1) \), а также производную функции \( u(M) \) в точке \( M_1 \) по направлению вектора \( \overrightarrow{M_1M_2} \):
\[
  u(M)=5xy^3z^3,\quad M_1(2,1,-1),\quad M_2(4,-3,0)
\]
\end{problem}

\begin{solution}
  По определению градиента функции \( u(x,y,z) \):
  %
  \[
    \begin{gathered}
      \grad u(x,y,z)=\left(\dfrac{\partial u}{\partial x},\dfrac{\partial u}{\partial y},\dfrac{\partial u}{\partial z}\right)=\left(5y^3z^3,15xy^2z^3,15xy^3z^2\right)\implies\\
      \implies\grad u(2,1,-1)=(-5,-30,30)
    \end{gathered}
  \]

  Определим нормированный вектор \( \gamma \):
  %
  \[
    \gamma=\dfrac{\overrightarrow{M_1M_2}}{\abs{\overrightarrow{M_1M_2}}}=\dfrac{(4-2,-3-1,0-(-1))}{\sqrt{4+16+1}}=\dfrac{1}{\sqrt{21}}(2,-4,1)
  \]

  По формуле связи производной функции \( u \) по направлению \( \overrightarrow{M_1M_2} \) с градиентом:
  %
  \[
    \dfrac{\partial u}{\partial\overrightarrow{M_1M_2}}=\dfrac{1}{\sqrt{21}}\grad u(2,1,-1)\cdot(2,-4,1)=\dfrac{1}{\sqrt{21}}(-5\cdot 2-30\cdot(-4)+30\cdot 1)=\dfrac{140}{\sqrt{21}}
  \]

  Итого:
  %
  \[
    \boxed{
      \grad u(2,1,-1)=(-5,-30,30),\quad\dfrac{\partial u}{\partial\overrightarrow{M_1M_2}}=\dfrac{140}{\sqrt{21}}
    }
  \]
\end{solution}