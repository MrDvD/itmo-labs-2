\documentclass[12pt]{article}
\usepackage[T1]{fontenc}
\usepackage[utf8]{inputenc}
\usepackage[english, russian]{babel}
\usepackage{amsmath, amssymb, amsthm}
\usepackage{graphicx}
\usepackage{float}
\usepackage{caption}
\usepackage{subcaption}
\usepackage{listings}
\usepackage{xcolor}
\usepackage{hyperref}
\usepackage[a4paper,
            left=3cm,
            right=3cm,
            top=2cm,
            bottom=3cm]{geometry}

\newtheorem{theorem}{Theorem}

% Настройки для листингов кода
\lstset{
    language=Python,
    basicstyle=\ttfamily\small,
    keywordstyle=\color{blue},
    commentstyle=\color{green!60!black},
    stringstyle=\color{red},
    numbers=left,
    numberstyle=\tiny,
    stepnumber=1,
    numbersep=5pt,
    frame=single,
    tabsize=4,
    breaklines=true,
    breakatwhitespace=true
}

% Параметры для гиперссылок
\hypersetup{
    colorlinks=true,
    linkcolor=blue,
    filecolor=magenta,      
    urlcolor=cyan,
    pdftitle={Отчет по фракталам}
}

\title{Отчёт по исследованию фракталов}
\author{Ваше Имя \\ Ваша Группа}
\date{\today}