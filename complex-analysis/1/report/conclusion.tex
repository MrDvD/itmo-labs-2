\section*{Заключение}

Были исследованы методы построения кривой Гильберта и показано, как простое рекурсивное правило приводит к образованию сложной, самоподобной структуры. Этот фрактал является классическим примером пространства-заполняющих кривых и демонстрирует идею предельного перехода от дискретных линий к непрерывной плоской форме.