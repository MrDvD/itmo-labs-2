\section{Кривая Гильберта}

\subsection{Описание структуры и построения}

\textbf{Кривая Гильберта} — это пространство-заполняющая кривая, впервые предложенная Давидом Гильбертом в 1891 году. Она является примером непрерывного отображения от единичного отрезка на единичный квадрат, при этом полностью заполняет его область. Кривая Гильберта строится итерационно, при каждой итерации увеличивая уровень вложенности и детализации.

\subsubsection{Алгоритм построения}

Построение кривой Гильберта основано на рекурсивной перестройке базового шаблона из четырёх подкрестов, соединённых под определённым углом. Каждый следующий порядок кривой делит квадрат на четыре меньших и встраивает в них повернутые копии кривой предыдущего порядка.

Кривую можно построить с помощью рекурсивной функции, описывающей направления движения пера:

\[
\text{Hilbert}(n, d) =
\begin{cases}
\text{влево},\, \text{Hilbert}(n-1, -d),\, \text{вперёд},\, \text{Направо}, \\
\text{Hilbert}(n-1, d),\, \text{вперёд},\, \text{Hilbert}(n-1, d), \\
\text{направо},\, \text{вперёд},\, \text{Hilbert}(n-1, -d)
\end{cases}
\]

где \( n \) — порядок кривой, а \( d \) определяет направление поворота.

\begin{lstlisting}[caption=Построение кривой Гильберта]
import matplotlib.pyplot as plt

def hilbert_curve(n, angle=90, step=10):
    path = []
    def hilbert(level, sign):
        if level == 0:
            return
        plt.right(sign * angle)
        hilbert(level - 1, -sign)
        plt.forward(step)
        plt.left(sign * angle)
        hilbert(level - 1, sign)
        plt.forward(step)
        hilbert(level - 1, sign)
        plt.left(sign * angle)
        plt.forward(step)
        hilbert(level - 1, -sign)
        plt.right(sign * angle)
    hilbert(n, 1)

import turtle as plt
plt.speed(0)
hilbert_curve(5, step=8)
plt.hideturtle()
plt.done()
\end{lstlisting}

\subsection{Визуализации}

\begin{figure}[H]
    \centering
    \begin{subfigure}{0.45\textwidth}
        % \includegraphics[width=\textwidth]{hilbert_1.png}
        \caption{Кривая Гильберта 1-го порядка}
    \end{subfigure}
    \hfill
    \begin{subfigure}{0.45\textwidth}
        % \includegraphics[width=\textwidth]{hilbert_3.png}
        \caption{Кривая Гильберта 3-го порядка}
    \end{subfigure}
    \\
    \begin{subfigure}{0.45\textwidth}
        % \includegraphics[width=\textwidth]{hilbert_5.png}
        \caption{Кривая Гильберта 5-го порядка}
    \end{subfigure}
    \hfill
    \begin{subfigure}{0.45\textwidth}
        % \includegraphics[width=\textwidth]{hilbert_zoom.png}
        \caption{Фрагмент с увеличением, показывающий самоподобие}
    \end{subfigure}
    \caption{Построения кривой Гильберта разных порядков}
\end{figure}

\subsection{Анализ структуры}

Кривая Гильберта обладает рядом интересных свойств:
\begin{itemize}
    \item \textbf{Самоподобие}: каждый уровень построения представляет собой масштабированную копию предыдущего, повернутую под разными углами.
    \item \textbf{Непрерывность}: кривая является непрерывной, но нигде не дифференцируемой функцией.
    \item \textbf{Заполнение пространства}: при бесконечном числе итераций кривая заполняет весь квадрат, переходя из одномерного пространства в двумерное.
\end{itemize}