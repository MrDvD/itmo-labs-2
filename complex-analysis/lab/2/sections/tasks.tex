\section{Постановка задач}

В ходе лабораторной работы \textit{по варианту №15} будут выполнены следующие задачи:

\begin{enumerate}
  \item Аналитически описать заданные множества.
  \item Воспользовавшись композицией классических преобразований, составить конформное отображение, которое переводит первую область во вторую.
  \item Составить обратное отображение, переводящее второе множество в первое.
  \item На любом удобном языке программирования написать программу, которая изобразит первое множество и все этапы его преобразования во второе. Достаточно наглядным будет взять набор точек множества, передающий его форму \textit{(может понадобится сделать набор «более плотным» в какой-то части множества)}.
\end{enumerate}

\begin{center}
  \begin{tblr}{
    colspec = {X[c] X[c]},
  }
  \begin{tikzpicture}[scale=0.8]

\def\radius{2.718}

% Create pattern on entire plane
\fill[pattern=north east lines, pattern color=black!30] 
    (-3.5,-3.5) rectangle (3.5,3.5);

% Remove pattern from a strip around negative real axis
\fill[white] (-3.5,-0.15) rectangle (0,0.15);

% Draw axes
\draw[->, thick] (-3.5,0) -- (3.5,0);
\draw[->, thick] (0,-3.5) -- (0,3.5);

\fill (0,0) circle (2.5pt) node[below right] {$0$};

\end{tikzpicture} &
\begin{tikzpicture}[scale=0.8]

\def\radius{2}

\begin{scope}
    \clip (-3.5,-3.5) rectangle (3.5,3.5);
    \fill[even odd rule, pattern=north east lines, pattern color=black!30] 
        (-3.5,-3.5) rectangle (3.5,3.5) (0,0) circle (\radius);
\end{scope}

\draw[very thick] (0,0) circle (\radius);

\draw[->, thick] (-3.5,0) -- (3.5,0);
\draw[->, thick] (0,-3.5) -- (0,3.5);

\fill (0,0) circle (2.5pt) node[below left] {$0$};
\fill (\radius,0) circle (2.5pt) node[below left] {$e$};

\end{tikzpicture} \\
  Рисунок 4 & Рисунок 9 \\
  \end{tblr}
\end{center}