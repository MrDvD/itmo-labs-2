\section{Аналитическое описание множеств}

Даны изображения множеств \( P \) и \( Q \) на комплексной плоскости \( \mathbb{C} \):
%
\begin{center}
  \begin{tblr}{
    colspec = {X[c] X[c]},
  }
  \input{tikz/P_plot} & \begin{tikzpicture}[scale=0.8]

\def\radius{2}

\begin{scope}
  \clip (-3.5,-3.5) rectangle (3.5,3.5);
  \fill[even odd rule, pattern=north east lines, pattern color=black!30] 
    (-3.5,-3.5) rectangle (3.5,3.5) (0,0) circle (\radius);
\end{scope}

\draw[dashed] (0,0) circle (\radius);

\draw[->, thick] (-3.5,0) -- (3.5,0) node[right] {$\Re z$};
\draw[->, thick] (0,-3.5) -- (0,3.5) node[above] {$\Im z$};

\fill (0,0) circle (2.5pt) node[below left] {$0$};
\fill (\radius,0) circle (2.5pt) node[below left] {$e$};

\end{tikzpicture} \\
  Множество \( P'\subset\mathbb{C} \) & Множество \( Q'\subset\mathbb{C} \) \\
  \end{tblr}
\end{center}

Видно, что множество \( P \) --- комплексная плоскость с разрезом по действительному лучу \( (-\infty,0] \). Множество \( Q \), в свою очередь, --- внешняя часть комплексной окружности радиуса \( e \):
%
\[
  P=\left\{z\in\mathbb{C}\,\middle|\,\Re z>0\lor\Im z\neq 0\right\}\qquad Q=\left\{z\in\mathbb{C}\,\middle|\,\abs{z}>e\right\}
\]