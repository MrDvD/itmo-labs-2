\section{Конформные отображения множеств}

Составим конформное отображение \( \omega(z) \), переводящее множество \( P \) в \( Q \):
%
\begin{enumerate}
  \item Применим к \( P \) конформное отображение \( \omega_1(z)=-z \):
  %
  \begin{center}
    \begin{tblr}{
      colspec = {X[c] X[c]},
    }
    \begin{tikzpicture}[scale=0.8]

  \fill[pattern=north east lines, pattern color=black!30] 
      (-3.5,-3.5) rectangle (3.5,3.5);

  \fill[white] (-3.5,-0.15) rectangle (0,0.15);

  \draw[->, thick] (-3.5,0) -- (3.5,0) node[right] {$\Re z$};
  \draw[->, thick] (0,-3.5) -- (0,3.5) node[above] {$\Im z$};

  \fill (0,0) circle (2.5pt) node[below right] {$0$};

  \end{tikzpicture} &
  \begin{tikzpicture}[scale=0.8]

  \fill[pattern=north east lines, pattern color=black!30] 
      (-3.5,-3.5) rectangle (3.5,3.5);

  \fill[white] (3.5,-0.15) rectangle (0,0.15);

  \draw[->, thick] (-3.5,0) -- (3.5,0) node[right] {$\Re z$};
  \draw[->, thick] (0,-3.5) -- (0,3.5) node[above] {$\Im z$};

  \fill (0,0) circle (2.5pt) node[below right] {$0$};

  \end{tikzpicture} \\
    Множество \( P \) & Множество \( P_1=\omega_1(P) \) \\
    \end{tblr}
  \end{center}

  \item Применим к \( P_1 \) конформное отображение \( \omega_2(z)=\sqrt{z} \):
  %
  \begin{center}
    \begin{tblr}{
      colspec = {X[c] X[c]},
    }
    \begin{tikzpicture}[scale=0.8]

  \fill[pattern=north east lines, pattern color=black!30] 
      (-3.5,-3.5) rectangle (3.5,3.5);

  \fill[white] (3.5,-0.15) rectangle (0,0.15);

  \draw[->, thick] (-3.5,0) -- (3.5,0) node[right] {$\Re z$};
  \draw[->, thick] (0,-3.5) -- (0,3.5) node[above] {$\Im z$};

  \fill (0,0) circle (2.5pt) node[below right] {$0$};

  \end{tikzpicture} &
  \begin{tikzpicture}[scale=0.8]

  \fill[pattern=north east lines, pattern color=black!30] 
      (-3.5,0) rectangle (3.5,3.5);

  \fill[white] (-3.5,-0.15) rectangle (3.5,0.15);

  \draw[->, thick] (-3.5,0) -- (3.5,0) node[right] {$\Re z$};
  \draw[->, thick] (0,-3.5) -- (0,3.5) node[above] {$\Im z$};

  \fill (0,0) circle (2.5pt) node[below right] {$0$};

  \end{tikzpicture} \\
    Множество \( P_1 \) & Множество \( P_2=\omega_2(P_1) \) \\
    \end{tblr}
  \end{center}

  \item Применим к \( P_2 \) конформное отображение \( \omega_3(z)=\frac{z-i}{z+i} \):
  %
  \begin{center}
    \begin{tblr}{
      colspec = {X[c] X[c]},
    }
    \begin{tikzpicture}[scale=0.8]

  \fill[pattern=north east lines, pattern color=black!30] 
      (-3.5,0) rectangle (3.5,3.5);

  \fill[white] (-3.5,-0.15) rectangle (3.5,0.15);

  \draw[->, thick] (-3.5,0) -- (3.5,0) node[right] {$\Re z$};
  \draw[->, thick] (0,-3.5) -- (0,3.5) node[above] {$\Im z$};

  \fill (0,0) circle (2.5pt) node[below right] {$0$};

  \end{tikzpicture} &
  \begin{tikzpicture}[scale=0.8]

\def\radius{1.2}

\begin{scope}
  \fill[even odd rule, pattern=north east lines, pattern color=black!30] 
      (0,0) circle (\radius);
\end{scope}

\draw[line width=6pt, white] (0,0) circle (\radius);
\draw[very thick] (0,0) circle (\radius);

\draw[->, thick] (-3.5,0) -- (3.5,0) node[right] {$\Re z$};
\draw[->, thick] (0,-3.5) -- (0,3.5) node[above] {$\Im z$};

\fill (0,0) circle (2.5pt) node[below left] {$0$};
\fill (\radius,0) circle (2.5pt) node[below left] {$1$};

\end{tikzpicture} \\
    Множество \( P_2 \) & Множество \( P_3=\omega_3(P_2) \) \\
    \end{tblr}
  \end{center}

  \item Применим к \( P_3 \) конформное отображение \( \omega_4(z)=\frac{1}{z} \):
  %
  \begin{center}
    \begin{tblr}{
      colspec = {X[c] X[c]},
    }
    \begin{tikzpicture}[scale=0.8]

\def\radius{1.2}

\begin{scope}
  \fill[even odd rule, pattern=north east lines, pattern color=black!30] 
      (0,0) circle (\radius);
\end{scope}

\draw[line width=6pt, white] (0,0) circle (\radius);
\draw[very thick] (0,0) circle (\radius);

\draw[->, thick] (-3.5,0) -- (3.5,0) node[right] {$\Re z$};
\draw[->, thick] (0,-3.5) -- (0,3.5) node[above] {$\Im z$};

\fill (0,0) circle (2.5pt) node[below left] {$0$};
\fill (\radius,0) circle (2.5pt) node[below left] {$1$};

\end{tikzpicture} &
  \begin{tikzpicture}[scale=0.8]

\def\radius{1.2}

\begin{scope}
  \fill[even odd rule, pattern=north east lines, pattern color=black!30] 
      (-3.5,-3.5) rectangle (3.5,3.5) (0,0) circle (\radius);
\end{scope}

\draw[line width=6pt, white] (0,0) circle (\radius);
\draw[very thick] (0,0) circle (\radius);

\draw[->, thick] (-3.5,0) -- (3.5,0) node[right] {$\Re z$};
\draw[->, thick] (0,-3.5) -- (0,3.5) node[above] {$\Im z$};

\fill (0,0) circle (2.5pt) node[below left] {$0$};
\fill (\radius,0) circle (2.5pt) node[below left] {$1$};

\end{tikzpicture} \\
    Множество \( P_3 \) & Множество \( P_4=\omega_4(P_3) \) \\
    \end{tblr}
  \end{center}

  \item Применим к \( P_4 \) конформное отображение \( \omega_5(z)=ez \):
  %
  \begin{center}
      \begin{tblr}{
        colspec = {X[c] X[c]},
      }
      \begin{tikzpicture}[scale=0.8]

  \def\radius{1.2}

  \begin{scope}
    \fill[even odd rule, pattern=north east lines, pattern color=black!30] 
        (-3.5,-3.5) rectangle (3.5,3.5) (0,0) circle (\radius);
  \end{scope}

  \draw[line width=6pt, white] (0,0) circle (\radius);
  \draw[very thick] (0,0) circle (\radius);

  \draw[->, thick] (-3.5,0) -- (3.5,0) node[right] {$\Re z$};
  \draw[->, thick] (0,-3.5) -- (0,3.5) node[above] {$\Im z$};

  \fill (0,0) circle (2.5pt) node[below left] {$0$};
  \fill (\radius,0) circle (2.5pt) node[below left] {$1$};

  \end{tikzpicture} &
    \begin{tikzpicture}[scale=0.8]

  \def\radius{2}

  \begin{scope}
    \fill[even odd rule, pattern=north east lines, pattern color=black!30] 
        (-3.5,-3.5) rectangle (3.5,3.5) (0,0) circle (\radius);
  \end{scope}

  \draw[line width=6pt, white] (0,0) circle (\radius);
  \draw[very thick] (0,0) circle (\radius);

  \draw[->, thick] (-3.5,0) -- (3.5,0) node[right] {$\Re z$};
  \draw[->, thick] (0,-3.5) -- (0,3.5) node[above] {$\Im z$};

  \fill (0,0) circle (2.5pt) node[below left] {$0$};
  \fill (\radius,0) circle (2.5pt) node[below left] {$e$};

  \end{tikzpicture} \\
      Множество \( P_4 \) & Множество \( P_5=\omega_5(P_4) \) \\
      \end{tblr}
  \end{center}
\end{enumerate}

Итого \( \omega(z) \) имеет вид:
%
\[ \omega=\omega_5\circ\omega_4\circ\omega_3\circ\omega_2\circ\omega_1\implies\omega(z)=e\cdot\dfrac{\sqrt{-z}+i}{\sqrt{-z}-i} \]

Составим обратное конформное отображение \( z(\omega) \), которое переводит множество \( Q \) в \( P \):

\begin{enumerate}
  \item Применим к \( Q \) конформное отображение \( z_1(\omega)=\frac{\omega}{e} \):
  %
\begin{center}
      \begin{tblr}{
        colspec = {X[c] X[c]},
      }
      \begin{tikzpicture}[scale=0.8]

  \def\radius{2}

  \begin{scope}
    \fill[even odd rule, pattern=north east lines, pattern color=black!30] 
        (-3.5,-3.5) rectangle (3.5,3.5) (0,0) circle (\radius);
  \end{scope}

  \draw[line width=6pt, white] (0,0) circle (\radius);
  \draw[very thick] (0,0) circle (\radius);

  \draw[->, thick] (-3.5,0) -- (3.5,0) node[right] {$\Re z$};
  \draw[->, thick] (0,-3.5) -- (0,3.5) node[above] {$\Im z$};

  \fill (0,0) circle (2.5pt) node[below left] {$0$};
  \fill (\radius,0) circle (2.5pt) node[below left] {$e$};

  \end{tikzpicture} &
    \begin{tikzpicture}[scale=0.8]

  \def\radius{1.2}

  \begin{scope}
    \fill[even odd rule, pattern=north east lines, pattern color=black!30] 
        (-3.5,-3.5) rectangle (3.5,3.5) (0,0) circle (\radius);
  \end{scope}

  \draw[line width=6pt, white] (0,0) circle (\radius);
  \draw[very thick] (0,0) circle (\radius);

  \draw[->, thick] (-3.5,0) -- (3.5,0) node[right] {$\Re z$};
  \draw[->, thick] (0,-3.5) -- (0,3.5) node[above] {$\Im z$};

  \fill (0,0) circle (2.5pt) node[below left] {$0$};
  \fill (\radius,0) circle (2.5pt) node[below left] {$1$};

  \end{tikzpicture} \\
      Множество \( Q \) & Множество \( Q_1=z_1(Q) \) \\
      \end{tblr}
  \end{center}

  \item Применим к \( Q_1 \) конформное отображение \( z_2(\omega)=\frac{1}{\omega} \):
  %
\begin{center}
    \begin{tblr}{
      colspec = {X[c] X[c]},
    }
    \begin{tikzpicture}[scale=0.8]

\def\radius{1.2}

\begin{scope}
  \fill[even odd rule, pattern=north east lines, pattern color=black!30] 
      (-3.5,-3.5) rectangle (3.5,3.5) (0,0) circle (\radius);
\end{scope}

\draw[line width=6pt, white] (0,0) circle (\radius);
\draw[very thick] (0,0) circle (\radius);

\draw[->, thick] (-3.5,0) -- (3.5,0) node[right] {$\Re z$};
\draw[->, thick] (0,-3.5) -- (0,3.5) node[above] {$\Im z$};

\fill (0,0) circle (2.5pt) node[below left] {$0$};
\fill (\radius,0) circle (2.5pt) node[below left] {$1$};

\end{tikzpicture} &
  \begin{tikzpicture}[scale=0.8]

\def\radius{1.2}

\begin{scope}
  \fill[even odd rule, pattern=north east lines, pattern color=black!30] 
      (0,0) circle (\radius);
\end{scope}

\draw[line width=6pt, white] (0,0) circle (\radius);
\draw[very thick] (0,0) circle (\radius);

\draw[->, thick] (-3.5,0) -- (3.5,0) node[right] {$\Re z$};
\draw[->, thick] (0,-3.5) -- (0,3.5) node[above] {$\Im z$};

\fill (0,0) circle (2.5pt) node[below left] {$0$};
\fill (\radius,0) circle (2.5pt) node[below left] {$1$};

\end{tikzpicture} \\
    Множество \( Q_1 \) & Множество \( Q_2=z_2(Q_1) \) \\
    \end{tblr}
  \end{center}

  \item Применим к \( Q_2 \) конформное отображение \( z_3(\omega)=i\cdot\frac{1+\omega}{1-\omega} \):
  %
\begin{center}
    \begin{tblr}{
      colspec = {X[c] X[c]},
    }
    \begin{tikzpicture}[scale=0.8]

\def\radius{1.2}

\begin{scope}
  \fill[even odd rule, pattern=north east lines, pattern color=black!30] 
      (0,0) circle (\radius);
\end{scope}

\draw[line width=6pt, white] (0,0) circle (\radius);
\draw[very thick] (0,0) circle (\radius);

\draw[->, thick] (-3.5,0) -- (3.5,0) node[right] {$\Re z$};
\draw[->, thick] (0,-3.5) -- (0,3.5) node[above] {$\Im z$};

\fill (0,0) circle (2.5pt) node[below left] {$0$};
\fill (\radius,0) circle (2.5pt) node[below left] {$1$};

\end{tikzpicture} &
  \begin{tikzpicture}[scale=0.8]

\def\radius{1.2}

\fill[pattern=north east lines, pattern color=black!30] 
      (-3.5,0) rectangle (3.5,3.5);

\fill[white] (-3.5,-0.15) rectangle (3.5,0.15);

\draw[->, thick] (-3.5,0) -- (3.5,0) node[right] {$\Re z$};
\draw[->, thick] (0,-3.5) -- (0,3.5) node[above] {$\Im z$};

\end{tikzpicture} \\
    Множество \( Q_2 \) & Множество \( Q_3=z_3(Q_2) \) \\
    \end{tblr}
  \end{center}

  \item Применим к \( Q_3 \) конформное отображение \( z_4(\omega)=\omega^2 \):
  %
\begin{center}
    \begin{tblr}{
      colspec = {X[c] X[c]},
    }
    \begin{tikzpicture}[scale=0.8]

\def\radius{1.2}

\fill[pattern=north east lines, pattern color=black!30] 
      (-3.5,0) rectangle (3.5,3.5);

\fill[white] (-3.5,-0.15) rectangle (3.5,0.15);

\draw[->, thick] (-3.5,0) -- (3.5,0) node[right] {$\Re z$};
\draw[->, thick] (0,-3.5) -- (0,3.5) node[above] {$\Im z$};

\end{tikzpicture} &
  \begin{tikzpicture}[scale=0.8]

\def\radius{1.2}

\fill[pattern=north east lines, pattern color=black!30] 
      (-3.5,-3.5) rectangle (3.5,3.5);

\fill[white] (0,-0.15) rectangle (3.5,0.15);

\draw[->, thick] (-3.5,0) -- (3.5,0) node[right] {$\Re z$};
\draw[->, thick] (0,-3.5) -- (0,3.5) node[above] {$\Im z$};

\end{tikzpicture} \\
    Множество \( Q_3 \) & Множество \( Q_4=z_4(Q_3) \) \\
    \end{tblr}
  \end{center}

  \item Применим к \( Q_4 \) конформное отображение \( z_5(\omega)=-\omega \):
  %
\begin{center}
    \begin{tblr}{
      colspec = {X[c] X[c]},
    }
    \begin{tikzpicture}[scale=0.8]

\def\radius{1.2}

\fill[pattern=north east lines, pattern color=black!30] 
      (-3.5,-3.5) rectangle (3.5,3.5);

\fill[white] (0,-0.15) rectangle (3.5,0.15);

\draw[->, thick] (-3.5,0) -- (3.5,0) node[right] {$\Re z$};
\draw[->, thick] (0,-3.5) -- (0,3.5) node[above] {$\Im z$};

\end{tikzpicture} &
  \begin{tikzpicture}[scale=0.8]

\def\radius{1.2}

\fill[pattern=north east lines, pattern color=black!30] 
      (-3.5,-3.5) rectangle (3.5,3.5);

\fill[white] (-3.5,-0.15) rectangle (0,0.15);

\draw[->, thick] (-3.5,0) -- (3.5,0) node[right] {$\Re z$};
\draw[->, thick] (0,-3.5) -- (0,3.5) node[above] {$\Im z$};

\end{tikzpicture} \\
    Множество \( Q_4 \) & Множество \( Q_5=z_5(Q_4) \) \\
    \end{tblr}
  \end{center}
\end{enumerate}

Итого \( z(\omega) \) имеет вид:
%
\[ z=z_5\circ z_4\circ z_3\circ z_2\circ z_1\implies z(\omega)=-\left(i\cdot\dfrac{1+\frac{e}{\omega}}{1-\frac{e}{\omega}}\right)^2 \]