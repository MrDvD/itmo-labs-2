\section{Заключение}

В ходе выполнения лабораторной работы была исследована задача конформного отображения заданных областей комплексной плоскости:

\begin{enumerate}
    \item Было дано аналитическое описание исходного множества \(P\) \textit{(комплексная плоскость с разрезом по действительному лучу \((-\infty, 0]\))} и целевого множества \(Q\) \textit{(внешность окружности радиуса \(e\))}.

    \item Построено конформное отображение 
    \[
    \omega(z) = e \cdot \frac{\sqrt{-z} + i}{\sqrt{-z} - i},
    \]
    представляющее собой композицию пяти элементарных преобразований:
    \begin{itemize}
        \item центральная симметрия,
        \item взятие квадратного корня,
        \item дробно-линейное преобразование,
        \item инверсия,
        \item растяжение на константу \(e\).
    \end{itemize}

    \item Построено обратное отображение 
    \[
    z(\omega) = -\left( i \cdot \frac{1 + \frac{e}{\omega}}{1 - \frac{e}{\omega}} \right)^2,
    \]
    переводящее множество \(Q\) обратно в \(P\).

    \item Реализован набор скриптов на Python, включающий:
    \begin{itemize}
        \item генерацию точек исходного множества с учётом его структуры и окраской в зависимости от аргумента и модуля,
        \item последовательное применение всех пяти отображений,
        \item визуализацию каждого этапа преобразования,
        \item экспорт данных в форматы, пригодные для построения графиков с помощью TikZ.
    \end{itemize}

    \item Получены наглядные графики всех промежуточных образов, подтверждающие корректность построенных отображений и сохранение конформности на каждом шаге.
\end{enumerate}

Работа демонстрирует практическое применение теории конформных отображений для преобразования областей комплексной плоскости, а также возможность автоматизации процесса визуализации с помощью скриптов.