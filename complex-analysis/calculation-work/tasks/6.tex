\begin{problem}
  Разложить функцию \( f(z) \) в ряд Лорана в указанной области:
  %
  \[ f(z)=\dfrac{3}{z^2-7z+10},\quad 2<\abs{z}<5 \]
\end{problem}

\begin{solution}
  Найдём нули знаменателя:
  %
  \[
    \begin{cases*}
      &z^2-7z+10=0\\
      &D=49-40=3^2
    \end{cases*}\implies
    \begin{sqcases*}
      &z=2\\
      &z=5
    \end{sqcases*}
  \]

  Разложим \( f(z) \) на простейшие дроби методом неопределённых коэффициентов:
  %
  \[
    \begin{gathered}
      \dfrac{A}{z-2}+\dfrac{B}{z-5}=\dfrac{3}{(z-2)(z-5)}\implies A(z-5)+B(z-2)=3\implies\\
      \implies\begin{cases*}
        &A+B=0\\
        &-5A-2B=3
      \end{cases*}\implies\begin{cases*}
        &A=-B\\
        &5B-2B=3
      \end{cases*}\implies\begin{cases*}
        &A=-1\\
        &B=1
      \end{cases*}
    \end{gathered}
  \]

  Разложим функцию \( f(z) \) в ряд Лорана в области \( 2<\abs{z}<5 \):
  %
  \[
    \begin{gathered}
      f(z)=\dfrac{1}{z-5}-\dfrac{1}{z-2}=-\dfrac{1}{5}\cdot\dfrac{1}{1-\frac{z}{5}}-\dfrac{1}{z}\cdot\dfrac{1}{1-\frac{2}{z}}=-\dfrac{1}{5}\sum_{n=0}^\infty\left(\dfrac{z}{5}\right)^n-\dfrac{1}{z}\sum_{n=0}^\infty\left(\dfrac{2}{z}\right)^n=\\
      =-\sum_{n=0}^\infty\dfrac{z^n}{5^{n+1}}-\sum_{n=0}^\infty\dfrac{2^n}{z^{n+1}}=-\sum_{-\infty}^{-1}\dfrac{z^n}{2^{n+1}}-\sum_{n=0}^{+\infty}\dfrac{z^n}{5^{n+1}}
    \end{gathered}
  \]

  Итого:
  %
  \[
    \boxed{
      f(z)=
      \sum_{-\infty}^{+\infty}c_n z^n,\quad c_n=\begin{cases*}
        &-2^{-n-1},\ n\leq -1\\
        &-5^{-n-1},\ n\geq 0
      \end{cases*}
    }
  \]
\end{solution}