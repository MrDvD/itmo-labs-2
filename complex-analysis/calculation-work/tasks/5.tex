\begin{problem}
  Разложить функцию \( f(z) \) в ряд Тейлора в окрестности точки \( z_0 \) и указать область, в которой ряд представляет данную функцию:
  %
  \[ f(z)=z^2 e^z,\quad z_0=1 \]
\end{problem}

\begin{solution}
  По определению ряда Тейлора в окрестности точки \( z_0=1 \):
  %
  \[
    f(z)=\sum_{n=0}^{\infty}\dfrac{f^{(n)}(1)}{n!}(z-1)^n
  \]

  По формуле Лейбница:
  %
  \[
    \begin{gathered}
      f^{(n)}=\sum_{k=0}^n\binom{n}{k}(z^2)^{(k)}(e^z)^{(n-k)}=\left(\binom{n}{0}z^2+\binom{n}{1}2z+\binom{n}{2}2\right)\cdot e^z=\\
      =(z^2+2nz+n(n-1))\cdot e^z\implies\\
      \implies f^{(n)}(1)=(1+2n+n^2-n)e=e(n^2+n+1)
    \end{gathered}
  \]

  Функция \( f(z) \) получена как произведение аналитичных функций на \( \mathbb{C} \), поэтому \( f(z) \) тоже аналитична на \( \mathbb{C} \):
  %
  \[
    \boxed{
      f(z)=e\sum_{n=0}^{\infty}\dfrac{n^2+n+1}{n!}(z-1)^n,\quad z\in\mathbb{C}
    }
  \]
\end{solution}